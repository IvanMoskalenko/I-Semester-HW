\section{Экспериментальное исследование}

Замеры производительности производились на ноутбуке под управлением операционной системы Fedora 33, с 16Гб оперативной памяти и процессором AMD Ryzen 7 4800H, 2.90GHz.  

В экспериментах использовались случайные списки и массивы, сгенерированные с использованием стандартного генератора случайных чисел \verb|System.Random|.
В качестве значений использовались 32-битные целые числа.

Эксперимент поставлен следующим образом. 
Для каждого значения длины 25 раз выполняются шаги, перечисленные ниже.
\begin{enumerate}
  \item Генерируется случайный список или массив заданной длины.
  \item Сгенерированный список или массив сортируется с использованием различных функций, при этом замеряется время, затраченное на сортировку.
\end{enumerate}

Полученные результаты записываются в файл.
Использован стандартный violin plot с указанием медианного значения времени для каждой длины.

Сначала мы провели сравнение всех тестируемых нами функций (кроме двух дополнительных вариаций \verb|qSort| для массива) со сборкой мусора и без~\ref{fig:GCvsNoGC} и выявили следующие закономерности.
\begin{enumerate}
  \item qSort~\subref{fig:list.qsortGCvsNoGC} и bSort~\subref{fig:list.bsortGCvsNoGC} для листа, а также qSort~\subref{fig:Array.qsortGCvsNoGC}, bSort~\subref{fig:Array.bsortGCvsNoGC} и System sort~\subref{fig:Array.sortGCvsNoGC} для массива ведут себя практически одинаково как со сборкой мусора, так и без неё.
  \item List.sort без сборки мусора, очевидно, работает медленнее~\subref{fig:list.sortGCvsNoGC}. Точное сравнение будет предоставлено в заключительной части доклада.   
\end{enumerate}
Затем мы провели сравнение всех тестируемых нами функций (кроме двух дополнительных вариаций \verb|qSort| для массива) в различных режимах сборки (Debug, Release)~\ref{fig:DebugVSRelease} и пришли к следующим выводам.
\begin{enumerate}
  \item Array.sort~\subref{fig:ArraySortDebugVSRelease} работает с одинаковой скоростью в различных режимах сборки.
  \item qSort~\subref{fig:ListQSortDebugVSRelease}, System sort~\subref{fig:ListSortDebugVSRelease} и bSort~\subref{fig:ListBubbleDebugVSRelease} для листа, а также qSort~\subref{fig:ArrayQSortDebugVSRelease} и bSort~\subref{fig:ArrayBubbleSortDebugVSRelease} для массива работают с разной скоростью. Во всех этих случаях результаты вычислялись быстрее в режиме сборки Release. Позже мы выясним, во сколько раз та или иная сортировка быстрее в режиме сборки Release.
\end{enumerate}
Потом мы провели нижеперечисленные действия.
\begin{enumerate}
  \item Сравнили \verb|qSort| с \verb|System sort| для массива~\subref{fig:ArraySortsComparisonSystemVSQSort}. Пришли к выводу, что системная сортировка быстрее.
  \item Сравнили \verb|qSort| с \verb|System sort| для листа~\subref{fig:ListSortsComparisonSystemVSQSort}. Пришли к выводу, что системная сортировка быстрее.
  \item Сравнили \verb|bSort| с \verb|qSort| и с \verb|System sort| для массива~\subref{fig:ArraySortsComparison}. Пришли к выводу, что сортировка ''пузырьком'' медленнее других.
  \item Сравнили \verb|bSort| с \verb|qSort| и с \verb|System sort| для листа~\subref{fig:ListSortsComparison}. Ситуация аналогична предыдущему пункту.
  \item Наконец, мы сравнили вариацию \verb|qSort| для массива с его первой неоптимизированной версией, которая создаёт дополнительные массивы во время своей работы~\subref{fig:ArrayQuickSortVSNotOptimizedQSort} и со второй~\subref{fig:ArraySortsComparisonSystemVSQSort}. Пришли к ожидаемому выводу: неоптимизированные версии медленнее. Подводя итоги, мы выясним причины такого поведения.
\end{enumerate}

\begin{figure}[H]
	\centering
	\begin{subfigure}[b]{0.40\textwidth}
       \centering
	   \includegraphics[width=0.99\textwidth]{data/ListSortGCvsNoGC.pdf}
	   \label{fig:list.sortGCvsNoGC}
	\end{subfigure}
	\begin{subfigure}[b]{0.40\textwidth}
       \centering
	   \includegraphics[width=0.99\textwidth]{data/ListQSortGCvsNoGC.pdf}
	   \label{fig:list.qsortGCvsNoGC}
	\end{subfigure}
	\begin{subfigure}[b]{0.40\textwidth}
       \centering
	   \includegraphics[width=0.99\textwidth]{data/ListBubbleSortGCvsNoGC.pdf}
	   \label{fig:list.bsortGCvsNoGC}
	\end{subfigure}
	\begin{subfigure}[b]{0.40\textwidth}
       \centering
	   \includegraphics[width=0.99\textwidth]{data/ArraySortGCvsNoGC.pdf}
	   \label{fig:Array.sortGCvsNoGC}
	\end{subfigure}
	\begin{subfigure}[b]{0.40\textwidth}
       \centering
	   \includegraphics[width=0.99\textwidth]{data/ArrayQSortGCvsNoGC.pdf}
	   \label{fig:Array.qsortGCvsNoGC}
	\end{subfigure}
	\begin{subfigure}[b]{0.40\textwidth}
       \centering
	   \includegraphics[width=0.99\textwidth]{data/ArrayBubbleSortGCvsNoGC.pdf}
	   \label{fig:Array.bsortGCvsNoGC}
	\end{subfigure}
	\caption{Сравнение производительности разных сортировок со сборкой мусора и без}
	\label{fig:GCvsNoGC}
\end{figure}

\begin{figure}[H]
	\centering
	\begin{subfigure}[b]{0.40\textwidth}
       \centering
	   \includegraphics[width=0.99\textwidth]{data/ListSortDebugVSRelease.pdf}
	   \label{fig:ListSortDebugVSRelease}
	\end{subfigure}
	\begin{subfigure}[b]{0.40\textwidth}
       \centering
	   \includegraphics[width=0.99\textwidth]{data/ListQSortDebugVSRelease.pdf}
	   \label{fig:ListQSortDebugVSRelease}
	\end{subfigure}
	\begin{subfigure}[b]{0.40\textwidth}
       \centering
	   \includegraphics[width=0.99\textwidth]{data/ListBubbleDebugVSRelease.pdf}
	   \label{fig:ListBubbleDebugVSRelease}
	\end{subfigure}
	\begin{subfigure}[b]{0.40\textwidth}
       \centering
	   \includegraphics[width=0.99\textwidth]{data/ArraySortDebugVSRelease.pdf}
	   \label{fig:ArraySortDebugVSRelease}
	\end{subfigure}
	\begin{subfigure}[b]{0.40\textwidth}
       \centering
	   \includegraphics[width=0.99\textwidth]{data/ArrayQSortDebugVSRelease.pdf}
	   \label{fig:ArrayQSortDebugVSRelease}
	\end{subfigure}
	\begin{subfigure}[b]{0.40\textwidth}
       \centering
	   \includegraphics[width=0.99\textwidth]{data/ArrayBubbleSortDebugVSRelease.pdf}
	   \label{fig:ArrayBubbleSortDebugVSRelease}
	\end{subfigure}
	\caption{Сравнение производительности разных сортировок в различных режимах конфигурации}
	\label{fig:DebugVSRelease}
\end{figure}

\begin{figure}[H]
	\centering
	\begin{subfigure}[b]{0.40\textwidth}
       \centering
	   \includegraphics[width=0.99\textwidth]{data/ListSortsComparison.pdf}
	   \label{fig:ListSortsComparison}
	\end{subfigure}
	\begin{subfigure}[b]{0.40\textwidth}
       \centering
	   \includegraphics[width=0.99\textwidth]{data/ArraySortsComparison.pdf}
	   \label{fig:ArraySortsComparison}
	\end{subfigure}
	\begin{subfigure}[b]{0.40\textwidth}
       \centering
	   \includegraphics[width=0.99\textwidth]{data/ArrayQuickSortVSNotOptimizedQSort.pdf}
	   \label{fig:ArrayQuickSortVSNotOptimizedQSort}
	\end{subfigure}
	\begin{subfigure}[b]{0.40\textwidth}
       \centering
	   \includegraphics[width=0.99\textwidth]{data/ListSortsComparisonSystemVSQSort.pdf}
	   \label{fig:ListSortsComparisonSystemVSQSort}
	\end{subfigure}
	\begin{subfigure}[b]{0.40\textwidth}
       \centering
	   \includegraphics[width=0.99\textwidth]{data/ArraySortsComparisonSystemVSQSort.pdf}
	   \label{fig:ArraySortsComparisonSystemVSQSort}
	\end{subfigure}
	\begin{subfigure}[b]{0.40\textwidth}
       \centering
	   \includegraphics[width=0.99\textwidth]{data/ArrayQuickSortVSNotOptimizedQuickSort_2.pdf}
	   \label{fig:ArraySortsComparisonSystemVSQSort}
	\end{subfigure}
	\caption{Сравнение производительности разных сортировок между собой}
	\label{fig:SortsComparison}	
\end{figure}



