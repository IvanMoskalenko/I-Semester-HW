\section{Введение}

Сортировка коллекций --- одна из фундаментальных алгоритмических задач, изучению которой посвящено множество работ.
На текущий момент разработано огромное количество алгоритмов сортировки.
Часть из них являются классическими, такие как сортировка пузырьком или сортировка Хоара (или быстрая сортировка), другие жё применяются в достаточно специфических областях.
Стоит также отметить, что некоторые сортировки представляют, в основном, академический интерес.
Например, сортировка пузырьком, применяемая, как правило, для образовательных целей.

Для того, чтобы обоснованно выбирать алгоритмы сортировки для тех или иных ситуаций, необходимо понимать не только их теоретические свойства, такие как теоретическую оценку сложности, но и то, как те или иные алгоритмы и их реализации, которые могут отличаться для одного и того же алгоритма, ведут себя на практике.

В данной работе будет проведено экспериментальное сравнение трёх сортировок для коллекции типа список (List) и четырёх сортировок для коллекции типа массив (Array) на платформе .NET.
А именно, мы сравним сортировки, предоставляемые стандартной библиотекой \verb|List.sort| и \verb|Array.sort|, а также \verb|Bubble Sort| и \verb|Quick Sort|. Для массива мы ещё сравним две вариации \verb|qSort|: одна из них использует память агрессивно.
